\documentclass[a4paper,12pt]{article}
\usepackage[portuguese]{babel}
\usepackage{graphicx, setspace, indentfirst}
\usepackage{amsmath}
\onehalfspace

\title{Notas de Estudo em Cálculo 3 e 4}
\author{Felipe Santos Araujo}
\date{June 2024}

\begin{document}

\maketitle

\newpage
\tableofcontents{}

\newpage
\section{Funções Vetoriais}
\subsection{Funções Vetoriais e Curvas Vetoriais}
    \newtheorem{definition}{Definição}
    Funções Vetoriais são funções que ao invés de lidarem com valores reais, lidam com vetores para descreverem curvas e superfícies no espaço.
        \begin{definition}
            Para todo número real \textit{t} existe um único vetor de $V_3$ denotado por $r(t)$. Seja $f(t)$, $g(t)$, $h(t)$ as componentes do vetor $r(t)$, então $f$, $g$ e $h$ são chamadas funções componentes de $r$ e podemos escrever:
            \[r(t) = \left<f(t), g(t), h(t) \right> = f(t)\hat{i} + g(t)\hat{j} + h(t)\hat{k}\]
        \end{definition}
    Note que a partir dessa definição podemos definir a parametrização de funções vetoriais como:
    \[x = f(t);     y = g(t);   z = h(t)\]
    Importante notar que o domínio dessa função $r$ é constituído por todos os valores de $t$ para os quais a expressão $r(t)$ está definida.\\

    O limite da função $r(t)$ pode ser definido como,
    \[\lim_{t \to a} r(t) = \left<\lim_{t \to a}f(t), \lim_{t \to a}g(t), \lim_{t \to a}h(t) \right>\]
    desde que $f$, $g$ e $h$ tenham limites existentes.\\
    \textbf{Exemplo 1} - Calcule
    \[\lim_{t \to 0} (e^{(-3t)}\hat{i} + \frac{t^2}{\sin^2{(t)}}\hat{j} + \cos{(2t)}\hat{k})\]
    Note que,
     \[\lim_{t \to 0} (e^{(-3t)}\hat{i} + \dfrac{t^2}{\sin^2{(t)}}\hat{j} + \cos{(2t)}\hat{k})\]
     \[= [\lim_{t \to 0}e^{(-3t)}]\hat{i} + [\lim_{t \to 0}\frac{t^2}{\sin^2{(t)}}]\hat{j} + [\lim_{t \to 0}\cos{(2t)}]\hat{k}\]
     \[= \hat{i} + \hat{j} + \hat{k}\]
\newpage
\subsection{Derivadas e Integrais de Funções Vetoriais}
    \subsubsection{Derivdas de Funções Vetoriais}
        \newtheorem{definition}{Definição}
            \begin{definition}
                A derivada de uma função vetorial $r$ é definida do mesmo modo como é feito para as funções de valores reais:
                \[\dfrac{dr}{dt} = r'(t) = \lim_{h \to 0} \frac{r(t + h) - r(t)}{h}\]
            \end{definition}
        \newtheorem{theorem}{Teorema}
            \begin{theorem}
                Seja \[r(t) = \left<f(t), g(t), h(t)\right> = f(t)\hat{i} + g(t)\hat{j} + h(t)\hat{k}\]
                Então, \[r'(t) = \left<f'(t), g'(t), h'(t)\right> = f'(t)\hat{i} + g'(t)\hat{j} + h'(t)\hat{k}\]
                Demonstração:
                \[r'(t) = \lim_{t \to 0} \frac{1}{\Delta t}[r(t + \Delta t) - r(t)]\]
                \[ = \lim_{t \to 0} \frac{1}{\Delta t}[\left<f(t + \Delta t), g(t + \Delta t), h(t + \Delta t)\right> - \left<f(t), g(t), h(t)\right>]\]
                \[ = \lim_{t \to 0} \left< \dfrac{f(t + \Delta t) - f(t))}{\Delta t}, \dfrac{g(t + \Delta t) - g(t))}{\Delta t}, \dfrac{h(t + \Delta t) - h(t))}{\Delta t}\right>\]
                \[ = \left< \lim_{t \to 0} \dfrac{f(t + \Delta t) - f(t))}{\Delta t}, \lim_{t \to 0} \dfrac{g(t + \Delta t) - g(t))}{\Delta t}, \lim_{t \to 0} \dfrac{h(t + \Delta t) - h(t))}{\Delta t}\right>\]
                \[= \left<f'(t), g'(t), h'(t)\right>\]
            \end{theorem}
    Observação: Todas as formas de derivação vistas para funções de valores reais servem também para funções vetoriais.
\newpage
    \subsubsection{Integral de Funções Vetoriais}

\end{document}
